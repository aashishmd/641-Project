\section{Other Problems}
\subsection{Other Problems 1}
\begin{frame}
One main contribution of FlumeJava is the optimization of execution plan by
reordering the data operations. For example, ParallelDo operations are fused,
related GroupByKey operations are combined by analyzing the data flow.
\begin{itemize}
  \item Correctness of the optimization is not described
  \item Same for many optimization techniques that is used in many frameworks
  seen in the course
  \item Also similar to the problem we picked, but more specific to single
  techniques in specific framework.
\end{itemize}
\end{frame}

\subsection{Other Problems 2}
\begin{frame}
A problem mentioned in the FlumeJava paper is that the abstraction of
operations make it easier for programmers to write MapReduce programs, but the
abstraction also causes programers to write program that is logical to them but
is inefficient. The inefficiency is caused by the lack of understanding of the
underlying execution.
\begin{itemize}
  \item This can be optimized by having a better analysis to remove redundant
  operations
  \item Very specific analysis for a specific framework and difficult to apply
  to other works.
\end{itemize}
\end{frame}

\subsection{Picked Problem}
\begin{frame}
\begin{itemize}
  \item The picked problem explores concurrent graph mutation in a concurrent context,
which can be applied to following works on concurrent graph processing
  \item Graph processing have always been an important class of problems that
  can be used applied to many real world problems
  \item Work on concurrent graph processing can be more broadly applied, and may
  be more interesting to broader audience.
\end{itemize}
\end{frame}