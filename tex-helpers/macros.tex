% $Id: macros.tex,v 1.4 2013/08/18 16:00:29 ganeshau Exp $
%\usepackage[pdftex]{hyperref}
\frenchspacing
\usepackage{mathptmx}

%\usepackage{cite}
\usepackage[T1]{fontenc}
\usepackage{times}
\usepackage{url}
%\usepackage{cite}
\usepackage{amsmath}
% \usepackage{fancyhdr}
%\usepackage{fancyvrb}
%\usepackage{fancybox}
\usepackage{color}
\usepackage{colortbl}
\usepackage{mathpartir}
\usepackage{amssymb}
\usepackage{xspace}
\usepackage{comment}
\usepackage{graphicx}
\graphicspath{{figures/}}
%\usepackage{epsfig}
%\usepackage{wrapfig}
\usepackage{multirow}
%\usepackage{subfig}

% Added listing for code listings
\usepackage{listings}
\usepackage{relsize}
%\usepackage{stfloats}
\usepackage{fixltx2e}
\usepackage{courier}
% formatting for grammars
% \input{obey}
% \input{grammar}

\usepackage{soul}
\newcommand{\hlc}[2][yellow]{ {\sethlcolor{#1} \hl{#2}} }

%\renewcommand{\nonterm}[1]{\mbox{\textit{#1}}}
\newcommand{\oneormore}[1]{#1\ensuremath{^+}}

% % %\newcommand{\panini}{\textsf{java}$_c$\xspace}
% \newcommand{\mechanism}{active module\xspace}
% \newcommand{\Mechanism}{Active module\xspace}
% \newcommand{\MEchanism}{Active Module\xspace}
% \newcommand{\mechanisms}{active modules\xspace}
% \newcommand{\Mechanisms}{Active modules\xspace}
% \newcommand{\MEchanisms}{Active Modules\xspace}
%\newcommand{\intermechanism}{inter-modular\xspace}
%\newcommand{\intramechanism}{intra-modular\xspace}

\newcommand{\mechanism}{capsule\xspace}
\newcommand{\Mechanism}{Capsule\xspace}
\newcommand{\MEchanism}{Calsupe\xspace}
\newcommand{\mechanisms}{capsules\xspace}
\newcommand{\Mechanisms}{Capsules\xspace}
\newcommand{\MEchanisms}{Capsules\xspace}
\newcommand{\intermechanism}{inter-capsular\xspace}
\newcommand{\intramechanism}{intra-capsular\xspace}

\newcommand{\MRB}{Message Ring Buffer\xspace}
\newcommand{\mrb}{message ring buffer\xspace}
\newcommand{\TaskFramework}{task framework\xspace}

% Type checking macros
% change symbol type of : from mathrel 
%\DeclareMathSymbol{:}{\mathbin}{operators}{"3A} 
\newcommand{\OK}{\mbox{OK}}
\newcommand{\OKin}{\mbox{OK in }}
\newcommand{\inC}{\mbox{in }c}
\newcommand{\isType}{\mbox{\textit{isType}}}
\newcommand{\validFields}{\mbox{\textit{validF}}}
\newcommand{\isClass}{\mbox{\textit{isClass}}}
\newcommand{\isHandler}{\mbox{\textit{isHandler}}}
\newcommand{\Types}{\mbox{\textit{Types}}}
\newcommand{\Names}{\mbox{\textit{Names}}}
\newcommand{\TypeEnv}{\mbox{\textit{TypeEnv}}}
\newcommand*{\findMeth}{\auxFunc{findMeth}}
\newcommand*{\fieldsOf}{\auxFunc{fields}}
\newcommand*{\fieldName}{\auxFunc{fieldName}}
\newcommand*{\checkMethods}{\auxFunc{methE}}
\newcommand*{\methodEffect}{\auxFunc{me}}
\newcommand*{\typeOfField}{\auxFunc{typeOfF}}
\newcommand*{\typeOf}{\auxFunc{typeOf}}
\newcommand*{\dom}{\auxFunc{dom}}
\newcommand*{\rng}{\auxFunc{rng}}
\newcommand*{\isAnn}{\auxFunc{annotated}}
\newcommand*{\updateEffect}{\auxFunc{update}}
\newcommand*{\reversePointer}{\auxFunc{reverse}}
\newcommand*{\reversePoint}{\auxFunc{reverseP}}
\newcommand*{\fixPoint}{\auxFunc{fixPoint}}
\newcommand*{\updateOther}{\auxFunc{updateOther}}
\newcommand*{\updateMethodEffect}{\auxFunc{updateEff}}
\newcommand*{\updateSingleMethodEffect}{\auxFunc{updateMeth}}
\newcommand*{\updateSingleEffect}{\auxFunc{concretize}}
\newcommand*{\updateForkEffect}{\auxFunc{updateFork}}
\newcommand*{\updateQueue}{\auxFunc{updateQ}}
\newcommand*{\getActive}{\auxFunc{active}\xspace}
\newcommand*{\field}{\auxFunc{field}}
\newcommand{\fdecl}{\mbox{\textit{fdecl}}}
\newcommand*{\fieldOf}{\auxFunc{fieldOf}}
\newcommand*{\getOpen}{\auxFunc{getOpen}}
\newcommand{\NPE}{\mbox{\textit{NPE}}}

% Reserved words (in math mode)
\newcommand{\Extends}{\mbox{\texttt{\textbf{extends}}}}
\newcommand{\Implements}{\mbox{\texttt{\textbf{implements}}}}
\newcommand{\This}{\mbox{\texttt{\textbf{this}}}}
\newcommand{\Self}{\mbox{\texttt{\textbf{self}}}}
\newcommand{\EV}{\mbox{\texttt{\textbf{event}}~}}
\newcommand{\Cast}{\mbox{\texttt{\textbf{cast}}}}
\newcommand{\New}{\mbox{\texttt{\textbf{new}}~}}
\newcommand{\Null}{\mbox{\texttt{\textbf{null}}}}
\newcommand{\Chain}{\mbox{\texttt{\textbf{chain}}~}}
\newcommand{\Under}{\mbox{\texttt{\textbf{under}}~}}
\newcommand{\If}{\mbox{\texttt{\textbf{if}}}}
\newcommand{\Then}{\mbox{\texttt{\textbf{then}}}}
\newcommand{\Else}{\mbox{\texttt{\textbf{else}}}}
\newcommand{\While}{\mbox{\texttt{\textbf{while}}}}

% Aux functions
\newcommand{\auxFunc}[1]{\ensuremath{\mathop{\mathit{#1}}}}
\newcommand*{\override}{\auxFunc{override}}

% types
\newcommand{\POWERSET}[1]{\mbox{\textit{PowerSet}}(#1)}
\newcommand{\delete}{\mbox{\textit{delete}}}
\newcommand{\mklist}{\mbox{\textit{mksupers}}}
\newcommand{\uminus}{\mbox{$\cup\!\!\!\!-$}}
\newcommand{\iminus}{\mbox{$\cap\!\!\!\!-$}}
\newcommand{\commutesWith}{\sharp}
\newcommand{\notCommutesWith}{\not\sharp}
\newcommand{\rname}[1]{$\TirName{(#1)}$} % for inline inferrule names
\newcommand{\STO}{\ensuremath{<:}}  % ``subtype of''
\newcommand{\notSTO}{\ensuremath{\not<:}}  % ``subtype of''
\newcommand{\consistent}{\ensuremath{\approx}}

% auxiliary funtions for finding the dynamic effects for program executions.
\newcommand*{\dynamicInfinite}{\auxFunc{dynValue}}
\newcommand*{\dynamicEffects}{\auxFunc{dynE}}
\newcommand*{\dynamicEffectsOnce}{\auxFunc{dyn}}
\newcommand*{\dynamicEffectSet}{\auxFunc{dynESet}}
\newcommand*{\predecessorSiblings}{\auxFunc{preS}}
\newcommand*{\predSibles}{\auxFunc{pSib}}
%\newcommand*{\offspring}{\auxFunc{descendentsOf}}
%\newcommand*{\offspring}{\auxFunc{descendents}}
\newcommand*{\offspring}{\auxFunc{desc}}
\newcommand*{\offspringOrItself}{\auxFunc{descendOrSelf}} 
\newcommand*{\inQueueX}{\auxFunc{inQ}} 
\newcommand*{\inQueueY}{\auxFunc{inq}}
\newcommand*{\replaceQueue}{\auxFunc{replace}}
\newcommand*{\substituteSequence}{\auxFunc{subSeq}}
\newcommand*{\combineSequence}{\auxFunc{comSeq}}
\newcommand*{\mergeSequence}{\auxFunc{merSeq}}
\newcommand*{\isin}{\auxFunc{belongs}}
\newcommand*{\configuration}{c}

% Macros for operational semantics
\newcommand{\mc}[1]{\mbox{\rm \ensuremath{\text{\code{#1}}}}} % Arg set in code font in math
\newcommand{\loc}{\ensuremath{\mathord{\mathit{loc}}}}
\newcommand{\handler}[3]{\ensuremath{\langle#1,#2,#3\rangle}}
\newcommand{\ec}{\ensuremath{\mathbb{E}}}
%\newcommand{\ec}{\ensuremath{\mathop{\mathbb{E}}}}
%\newcommand{\ecpending}{\ensuremath{\mathop{\mathbb{E'}}}}
\newcommand{\ecpending}{\ensuremath{\mathop{\overline{\mathbb{E}}}}}
\newcommand{\hole}{\ensuremath{\mathord{\mathit{-}}}}
\newcommand{\reducesto}{\hookrightarrow}
\newcommand{\reducestostar}{\overset{*}{\reducesto}}
\newcommand{\dontcare}{\ensuremath{\mathord{\text{\textvisiblespace\hspace{0.10ex}}}}}
\newcommand{\Expression}{\ensuremath{{\cal E}}}
\newcommand{\Stack}{\mbox{\textit{Stack}}}
\newcommand{\Store}{\mbox{\textit{Store}}}
\newcommand{\ActiveList}{\mbox{\textit{ActiveList}}}
\newcommand{\Frame}{\mbox{\textit{Frame}}}
\newcommand{\FieldEnv}{\mbox{\textit{FieldEnv}}}
\newcommand{\ObjectRecord}{\mbox{\textit{ObjectRecord}}}
\newcommand{\lexframe}{\mbox{\textbf{\texttt{frame}}}}
\newcommand{\handlerframe}{\mbox{\textbf{\texttt{hframe}}}}
\newcommand{\Excep}{\mbox{\textit{Excep}}}

\newcommand*{\seq}[1]{\ensuremath{\left\langle {#1} \right\rangle}} % Arg is sequence contents
\newcommand{\config}{\seq}
\newcommand{\udot}{\mathbin{\sqcup \kern-0.53em \cdot \,}}

\newcommand{\eseq}{\hat{e}}

% Effect attributes
\newcommand{\ReadE}{\mbox{\texttt{\textbf{read}}~}}
%\newcommand{\WriteE}{\mbox{\texttt{\textbf{write}}~}}
\newcommand{\WriteE}{\mbox{\texttt{\textbf{write}}~}}
\newcommand{\CheckE}{\mbox{\texttt{\textbf{open}}}}
%\newcommand{\ForkE}{\mbox{\texttt{\textbf{fork}}}}
\newcommand{\ForkE}{\mbox{\texttt{\textbf{fork}}}}
\newcommand{\EnqueueE}{\mbox{\texttt{\textbf{bottom}}}}
%\newcommand{\BottomE}{\mbox{\texttt{\textbf{\bot}}}}
\newcommand{\RegisterE}{\mbox{\texttt{\textbf{exp}}}}
%\newcommand{\AnnounceE}{\mbox{\texttt{\textbf{frk}}~}}
\newcommand{\CreateE}{\mbox{\texttt{\textbf{create}}~}}
% Effect Auxiliary Functions
\newcommand*{\EffectsOf}{\auxFunc{effects}}
\newcommand{\ReadF}{\mbox{\texttt{\textbf{rd}}}}
\newcommand{\WriteF}{\mbox{\texttt{\textbf{wt}}}}
\newcommand{\ForkF}{\mbox{\texttt{\textbf{fk}}}}
\newcommand{\EnqueueF}{\mbox{\texttt{\textbf{eq}}}}
\newcommand{\BottomF}{\mbox{\texttt{\textbf{bt}}}}
\newcommand{\ConcretizedF}{\mbox{\texttt{\textbf{ct}}}}
\newcommand{\OtherF}{\mbox{\texttt{\textbf{el}}}}


\newcommand{\Class}{\mbox{\texttt{\textbf{class}}~}}
\newcommand{\tasks}{\mbox{\texttt{\textbf{tasks}}~}}
\newcommand{\fork}{\mbox{\texttt{\textbf{fork}}~}}
\newcommand{\foreach}{\mbox{\texttt{\textbf{forall}}~}}
\newcommand{\Decl}{\mbox{\texttt{\textbf{decl}}~}}
\newcommand{\RDecl}{\mbox{\texttt{\textbf{rdecl}}~}}
\newcommand{\Exp}{\mbox{\texttt{\textbf{exp}}~}}
\newcommand{\Prog}{\mbox{\texttt{\textbf{prog}}~}}
\newcommand{\Meth}{\mbox{\texttt{\textbf{meth}}~}}
\newcommand{\Type}{\mbox{\texttt{\textbf{type}}}}
\newcommand{\var}{\mbox{\textit{var}}}
\newcommand{\Nil}{\ensuremath{\bullet}}
%\newcommand{\st}{\ensuremath{\textrm{s.t.}}}
\newcommand{\true}{\ensuremath{\mathit{true}}}
\newcommand{\false}{\ensuremath{\mathit{false}}}
\newcommand{\Land}{\ensuremath{\bigwedge}}

% New Panini features
\newcommand{\Announce}{\mbox{\texttt{\textbf{announce}}~}}
\newcommand{\Yield}{\mbox{\texttt{\textbf{yield}}~}}
%\newcommand{\Future}{\mbox{\texttt{\textbf{Future}}~}}

\newcommand{\When}{\mbox{\texttt{\textbf{when}}}}
\newcommand{\Do}{\mbox{\texttt{\textbf{do}}}}
\newcommand{\Return}{\mbox{\texttt{\textbf{return}}}}
\newcommand{\Invoke}{\mbox{\texttt{\textbf{invoke}}}}
\newcommand{\Thunk}{\mbox{\texttt{\textbf{thunk}}~}}
\newcommand{\Register}{\mbox{\texttt{\textbf{register}}}}
\newcommand{\EventClosure}{\mbox{\texttt{\textbf{eClosure}}}}
\newcommand{\Boolean}{\mbox{\texttt{\textbf{boolean}}}}
\newcommand{\Int}{\mbox{\texttt{\textbf{int}}}}
\newcommand{\Void}{\mbox{\texttt{\textbf{void}}}}
\newcommand{\Nat}{\mbox{\texttt{\textbf{Nat}}}}

% auxiliary funtions for updating the event hierarchies
%\newcommand*{\updateGamma}{\auxFunc{updateHierarchy}}
\newcommand*{\updateGamma}{\auxFunc{updateHier}}
\newcommand*{\inGamma}{\auxFunc{registered}}
\newcommand*{\inDelta}{\auxFunc{inHierarchy}}
\newcommand*{\inZeta}{\auxFunc{inList}}
%\newcommand*{\updateEffect}{\auxFunc{updateEffect}}
\newcommand*{\collectBindings}{\auxFunc{colEvts}}
\newcommand*{\collectEvents}{\auxFunc{colEvt}}
\newcommand*{\putEvent}{\auxFunc{putEvent}}
\newcommand*{\putHierarchies}{\auxFunc{putHierarchies}}
\newcommand*{\putLevel}{\auxFunc{putLevel}}
\newcommand*{\effectsHierachy}{\auxFunc{effHier}}
\newcommand*{\effectsList}{\auxFunc{effList}}
\newcommand*{\effectsUnion}{\auxFunc{effUn}}
\newcommand*{\mergeRho}{\auxFunc{mergeRho}}
\newcommand*{\mergeEpsilon}{\auxFunc{mergeEff}}
\newcommand*{\effectEnlarge}{\auxFunc{effectEnlarge}}
\newcommand*{\insertLast}{\auxFunc{insert}}
\newcommand*{\compareLast}{\auxFunc{comp}}
\newcommand*{\build}{\auxFunc{build}}
\newcommand*{\isConcurrent}{\auxFunc{parallel}}
\newcommand*{\conflict}{\auxFunc{disjointMeth}}
\newcommand*{\disjoint}{\auxFunc{disj}}
\newcommand*{\interfere}{\auxFunc{nonConfl}}
\newcommand*{\deepCopy}{\auxFunc{copy}}
\newcommand*{\findMax}{\auxFunc{fresh}}
\newcommand*{\forkTask}{\auxFunc{buildconfs}}
\newcommand*{\intersect}{\auxFunc{intersect}}
\newcommand*{\inQueue}{\auxFunc{inQueue}}
\newcommand*{\copyEffect}{\auxFunc{cp}}
\newcommand*{\getTask}{\auxFunc{getTask}}
\newcommand*{\getFirstTask}{\auxFunc{getfirst}}

\definecolor{Brown}{cmyk}{0,0.81,1,0.60}
\definecolor{OliveGreen}{cmyk}{0.64,0,0.95,0.40}
\definecolor{CadetBlue}{cmyk}{0.62,0.57,0.23,0}
\definecolor{lightlightgray}{gray}{0.9}

\definecolor{lightest-gray}{rgb}{0.95,0.95,0.5}
\definecolor{lighter-gray}{rgb}{0.7,0.98,0.7}
\definecolor{light-gray}{rgb}{0.5,0.98,0.98}
%\definecolor{light-gray}{rgb}{0.0,0.75,0.80}
\definecolor{dark-gray}{rgb}{0.95,0.7,0.95}
\definecolor{darker-gray}{rgb}{1,0.65,0.65}

% settings for listings
\definecolor{white}{gray}{1.0}
\definecolor{lightergray}{gray}{0.99}
\definecolor{lightgray}{gray}{0.97}
\definecolor{darkgray}{gray}{0.5}
\definecolor{OliveGreen}{cmyk}{0.64,0,0.95,0.40}

\lstset{
	language={[Visual]Basic}, emph={},
	mathescape=false, escapechar=@,
	backgroundcolor=\color{lightgray},
	commentstyle=\color{darkgray},
	keywordstyle=\color{OliveGreen}\bfseries,
	basicstyle=\scriptsize\sffamily,
	numberstyle=\scriptsize\sffamily,
	emphstyle=\bfseries,
	numbers=left, stepnumber=1,
	numberblanklines=false,
	numberstyle=\tiny,
	numbersep=-3pt,
	frame=none, framexleftmargin=0pt, framexrightmargin=0pt, 
	%xleftmargin=15pt, xrightmargin=4pt,
	columns=flexible, breaklines=true,
	showspaces=false, showstringspaces=false, showtabs=false, tabsize=2,
	morekeywords={abstract,case,catch,class,def,do,else,extends,%
          false,final,finally,for,forSome,if,implicit,import,lazy,%
          match,new,null,object,override,package,private,protected,%
          return,sealed,super,this,throw,trait,true,try,type,%
          val,var,while,with,yield,
          map,filter,flatmap,sample,groupByKey,reduceByKey,
          union,join,cogroup,crossProduct,mapValues,sort,partitionBy,
          count,collect,reduce,lookup,save},
	otherkeywords={=>,<-,<\%,<:,>:,\#,@}
}

% could use \relsize{-2} instead of \scriptsize below
\newcommand{\FIGCODEFONT}{\relsize{-1}\ttfamily}
\newcommand{\FIGCODELIBFONT}{\relsize{-6}\ttfamily}

% cross referencing
\newcommand{\figref}[1]{Figure~\ref{#1}}
\newcommand{\tabref}[1]{Table~\ref{#1}}
\newcommand{\fignref}[1]{Figure~\ref{#1}}
\newcommand{\appref}[1]{Appendix~\ref{#1}}
\newcommand{\secref}[1]{\S\ref{#1}}
\newcommand{\secnref}[1]{\S\ref{#1}}
\newcommand{\defref}[1]{Definition~\ref{#1}}
\newcommand{\lemref}[1]{Lemma~\ref{#1}}
\newcommand{\theref}[1]{Theorem~\ref{#1}}
\newcommand{\lineref}[1]{line~\ref{#1}}
\newcommand{\linesref}[2]{lines~\ref{#1}-\ref{#2}}
\newcommand{\linerefNum}[1]{\ref{#1}}
\newcommand{\footnoteref}[1]{$^{\ref{#1}}$}

%  % {theorems}
%  \newtheorem{theorem}{Theorem}[section]
%  \newtheorem{axiom}[theorem]{Axiom}
%  \newtheorem{corollary}[theorem]{Corollary}
%  \newtheorem{definition}[theorem]{Definition}
%  \newtheorem{example}[theorem]{Example}
%  \newtheorem{fact}[theorem]{Fact}
%  \newtheorem{lemma}[theorem]{Lemma}
%  \newtheorem{proposition}[theorem]{Proposition}
%  \newtheorem{remark}[theorem]{Remark}
%  \newtheorem{conjecture}[theorem]{Conjecture}

%Some helpful notation
\newcommand{\PROOF}{{\em Proof:\/}~~}

\newcommand{\etal}{~\textit{et al.}}

%\newcommand{\PROOFSKETCH}{{\em Proof Sketch:\/}~~}
\newcommand{\PROOFSKETCH}{{\em Proof:\/}~~}
%\newcommand{\QED}{\rule{0.4em}{0.65em}}

\newcommand\para[1]{\vspace{0.1em}{\bf #1.}\ }
%\newcommand\para[1]{\paragraph{#1}}

% formatting of initial section quotations
\newcommand{\QUOTATION}[1]{\begin{scriptsize}\begin{flushright}\emph{#1}\end{flushright}\end{scriptsize}}

% \newtheorem{theorem}{{\bf Theorem}}[section]
% \newenvironment{definition}[1][Definition]{\begin{trivlist}
% \item[\hskip \labelsep {\bfseries #1}]}{\end{trivlist}}
% 
% \newenvironment{lemma}[1][Lemma]{\begin{trivlist}
% \item[\hskip \labelsep {\bfseries #1}]}{\end{trivlist}}
% 
% \newenvironment{example}[1][Example]{\begin{trivlist}
% \item[\hskip \labelsep {\bfseries #1}]}{\end{trivlist}}
% 
% \newenvironment{remark}[1][Remark]{\begin{trivlist}
% \item[\hskip \labelsep {\bfseries #1}]}{\end{trivlist}}

% \newcommand{\qed}{\nobreak \ifvmode \relax \else
%       \ifdim\lastskip<1.5em \hskip-\lastskip
%       \hskip1.5em plus0em minus0.5em \fi \nobreak
%       \vrule height0.75em width0.5em depth0.25em\fi}

% Allow figures to take up the entire page
\renewcommand\floatpagefraction{.99}
\renewcommand\topfraction{.99}
\renewcommand\bottomfraction{.90}
\renewcommand\textfraction{.01}

% \renewcommand{\topfraction}{0.90}
% \renewcommand{\textfraction}{0.05}
% \renewcommand{\floatpagefraction}{0.85}
